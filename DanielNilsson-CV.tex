%% start of file `template_en.tex'.
%% Copyright 2006-1008 Xavier Danaux (xdanaux@gmail.com).
%
% This work may be distributed and/or modified under the
% conditions of the LaTeX Project Public License version 1.3c,
% available at http://www.latex-project.org/lppl/.


\documentclass[11pt,a4paper]{moderncv}

\usepackage{skills}

% moderncv themes
%\moderncvtheme[blue]{casual}      % optional argument are 'blue' (default),
                                   % 'orange', 'red', 'green', 'grey' and 'roman'
                                   % (for roman font)
\moderncvtheme[blue]{classic}

% character encoding
\usepackage[utf8]{inputenc}

% adjust the page margins
\usepackage[scale=0.8]{geometry}
%\setlength{\hintscolumnwidth}{3cm}	     % width of the column with the dates


%\AtBeginDocument{\setlength{\maketitlenamewidth}{6cm}}  % only for the classic
                                                         % theme, if you want to
                                                      % change the width of
                                                         % your name placeholder
\AtBeginDocument{\recomputelengths}

% personal data
\firstname{Daniel}
\familyname{Nilsson}
%\title{\small{Software Engineering student}}               % optional
\address{Hällskriftsgatan 8d}{417 26 Göteborg}    % optional
\mobile{0735 25 66 98}                           % optional
\email{deinils@gmail.com}                     % optional
\extrainfo{}  % optional
\photo[64pt]{picture}                            % optional

%\nopagenumbers{}


%-------------------------------------------------------------------------------
%            content
%-------------------------------------------------------------------------------
\begin{document}
\maketitle

\section{Utbildning}
\cventry{2007--nuvarande}{Software Engineering and Management}
{Göteborgs Universitet, IT University of Göteborg}{}{}{3-årigt
  kandidatprogram. Examen i maj 2010.}

\cventry{2001--2002}{Räddningsman/amröjare}
{Civilplikt}{}{}{}

\cventry{1999--2001}{Samhälle/ekonomi}
{Polhemsgymnasiet}{}{}{}

\section{Yrkeserfarenhet}
\cventry{2010--nuvarande}{Butiksbiträde} {Coop
  Bäckebol}{Göteborg}{}{Arbetar extra i kassa och som säljare vid
  sidan av studierna.}

\cventry{2009--2010}{Handledare} {Göteborgs
  Universitet}{Göteborg}{}{Handleder och stöttar elever i
  kurserna Software Architectures och Project: Software Architecture
  for Distributed Systems.  Arbetet innebär att pedagogiskt kunna
  förklara teori och begrepp inom ämnet samt hjälpa med praktiska
  problem i skapandet av en mjukvaruarkitektur. Som handledare rättar
  jag även inlämningsuppgifter och lämnar konstruktiv feedback till
  eleverna.}

\cventry{2006--2007}{V-Laget}
{Volvo Lastvagnar AB}{Göteborg}{}{Arbetade inom en mindre grupp som ansvarade för monteringen av lastbilar som avvek från normal
monterings process. Inom gruppen planerade vi det dagliga arbetet själva och fick ta ett större
ansvar för kvalitét och leverans.}

\cventry{2004--2006}{Gruppombud \& kontrollant} {Volvo Lastvagnar
  AB}{Göteborg}{}{Som gruppombud ansvarade jag för det dagliga arbetet
  på avdelningen som bland annat innebar delegering av
  arbetsuppgifter, utbildning av personal och kommunikation mellan
  avdelningar. Som gruppombud krävdes det att jag var stresstålig och
  kunde hantera situationer som inträffade både med produktion och
  personal. Som kontrollant var jag ytterst ansvarig för kvailtéten
  och säkerheten på produkten vi levererade vilket kräver nogrannhet
  och tålamod och färdighet inom kommunikation då felåterföring är en
  stor del av kvailitétsförbättrings processen.}

\cventry{2002--2004}{Montör} {Volvo Lastvagnar
  AB}{Göteborg}{}{Montering av lastbil.}

\section{Tekninsk kunskap}
\cvitem{Språk}   {C, Erlang, Java, Python, \LaTeX}
\cvitem{Applikationer}    {Emacs, Eclipse, SVN, Git, Microsoft Office, MySql}
\cvitem{System}     {GNU/Linux, Windows}
\cvitem{API's}       {JSP}
\cvitem{Koncept}    {OO, UML, Quality Assurance, Agile development,
  Change Management}

\section{Språk}
\cvlanguage{Svenska}    {Modersmål}{}
\cvlanguage{Engelska}    {Flytande i tal och skrift}{}


\renewcommand{\listitemsymbol}{\textbullet} % change the symbol for lists

\section{Övriga uppdrag}
\cvlistitem{\textbf{Aktiv medlem i förening GÖSTA 2009}\\
GÖSTA är en förening startad av eleverna på IT-Universitetet och annordnar varje år sveriges största IT-Arbetsmarknads dag.
Jag arbetade med att sälja platser till mässan mot företag och vara kontaktperson vid mässtillfället.}

\section{Personlig referens}
\cvitem{Namn}   {Rodrigo Fierro}
\cvitem{Yrke}   {Produktionsledare Volvo Lastvagnar}
\cvitem{Telefon}   {031 660 000 (växel)}
%\cvitem{E-post}   {henrik.sandklef@ituniv.se}

\section{Övrig information}
\cvlistitem{Innehar B-Körkort.} \\
\cvlistitem{Ytterligare referenser och orginalbetyg lämnas mot begäran.} \\

\end{document}


%% end of file `template_en.tex'.
